% CMOR 462 Project Report - Minimum Risk Portfolios
% Authors: Teymour Davoudi and Patrick Batsell
% Date: December 2025

\documentclass[11pt,letterpaper]{article}

% Packages
\usepackage[margin=1in]{geometry}
\usepackage{amsmath,amssymb}
\usepackage{graphicx}
\usepackage{booktabs}
\usepackage{float}
\usepackage{caption}
\usepackage{subcaption}
\usepackage{hyperref}
\usepackage{natbib}
\usepackage{setspace}
\usepackage{multirow}
\usepackage{array}
\usepackage{longtable}

% Formatting
\onehalfspacing
\setlength{\parindent}{0pt}
\setlength{\parskip}{6pt}

% Title information
\title{\textbf{Minimum Risk Portfolios: A Comparative Analysis of Covariance Estimation Methods}}
\author{
    Teymour Davoudi \and Patrick Batsell \\
    \textit{Rice University} \\
    \textit{CMOR 462: Optimization Methods in Finance}
}
\date{December 2025}

\begin{document}

\maketitle

\begin{abstract}
This paper investigates the performance of minimum variance portfolios constructed using different covariance estimation methods. We compare three approaches---the single-factor model, constant correlation model, and shrinkage estimator---against equal-weighted and value-weighted benchmarks. Using monthly data from the CRSP database spanning January 2017 to December 2022, we implement a rolling-window backtesting framework with 60-month estimation periods. Additionally, we examine the impact of imposing a 3\% upper bound constraint on individual stock weights to prevent portfolio concentration. Our results show that minimum variance portfolios generally achieve lower volatility than benchmarks, with the shrinkage estimator demonstrating the most favorable risk-adjusted performance. The 3\% upper bound constraint enhances diversification but increases turnover and slightly reduces risk reduction benefits.
\end{abstract}

\section{Introduction}

Since Markowitz's \citeyearpar{markowitz1952} seminal work on portfolio selection, the mean-variance framework has formed the cornerstone of modern portfolio theory. While expected return maximization attracts significant attention, the minimum variance portfolio---the leftmost point on the efficient frontier---offers a purely risk-focused investment strategy that requires no expected return estimates. This feature makes minimum variance portfolios particularly attractive for practical implementation, as expected returns are notoriously difficult to estimate accurately \citep{merton1980}.

However, constructing minimum variance portfolios in practice presents a significant challenge: estimating the covariance matrix of asset returns. With $N$ assets, the covariance matrix contains $N(N+1)/2$ unique parameters to estimate. When $N$ is large relative to the sample size $T$, the sample covariance matrix becomes ill-conditioned and unstable, leading to extreme and unstable portfolio weights \citep{jobson1980}. This estimation error can severely degrade out-of-sample performance.

\subsection{Research Questions}

This paper addresses three primary research questions:

\begin{enumerate}
    \item Which covariance estimation method---single-factor model, constant correlation model, or shrinkage estimator---produces the best out-of-sample risk-adjusted returns for minimum variance portfolios?
    \item How do minimum variance portfolios compare to naive diversification strategies (equal-weighted and value-weighted portfolios) in terms of risk reduction and risk-adjusted performance?
    \item What is the impact of imposing upper bound constraints (e.g., 3\% maximum weight per stock) on portfolio concentration, turnover, and performance?
\end{enumerate}

\subsection{Contribution}

Our analysis contributes to the literature by providing an empirical comparison of covariance estimation methods in a recent market period (2017-2022) that includes significant volatility events such as the COVID-19 pandemic. We extend previous work by \cite{clarke2006} by explicitly analyzing the impact of upper bound constraints on portfolio characteristics and by examining performance during a period of elevated market stress.

\section{Methodology}

\subsection{Data Description}

We use monthly stock return and market capitalization data from the Center for Research in Security Prices (CRSP) database. Our sample period spans from January 2012 to December 2022, with the out-of-sample testing period running from January 2017 to December 2022. The earlier period (2012-2016) serves as the initial estimation window.

Our universe consists of all common stocks with available return and market capitalization data in each month, typically ranging from 200 to 400 stocks after filtering. We use the CRSP value-weighted market return as our market proxy and the 90-day Treasury bill rate as the risk-free rate.

\subsection{Covariance Estimation Methods}

We implement three structured covariance estimators designed to improve stability and out-of-sample performance relative to the sample covariance matrix.

\subsubsection{Single-Factor Model}

The single-factor (market) model decomposes stock returns into a systematic component driven by the market and an idiosyncratic component:

\begin{equation}
R_i = \alpha_i + \beta_i R_m + \epsilon_i
\end{equation}

where $R_i$ is the return on stock $i$, $R_m$ is the market return, $\beta_i$ is the stock's market beta, and $\epsilon_i$ is the idiosyncratic return with $E[\epsilon_i] = 0$ and $\text{Cov}(\epsilon_i, R_m) = 0$.

The covariance matrix under this model has the form:

\begin{equation}
\Sigma_{\text{SF}} = \beta \beta' \sigma_m^2 + D
\end{equation}

where $\beta$ is the $N \times 1$ vector of market betas, $\sigma_m^2$ is the variance of market returns, and $D = \text{diag}(\sigma_{\epsilon_1}^2, \ldots, \sigma_{\epsilon_N}^2)$ is the diagonal matrix of residual variances.

This model reduces the number of parameters from $N(N+1)/2$ to $2N+1$, substantially improving estimation stability when $N$ is large relative to $T$.

\subsubsection{Constant Correlation Model}

The constant correlation model assumes all pairwise correlations are equal to the average correlation \citep{elton1973}:

\begin{equation}
\rho_{ij} = \bar{\rho} \quad \text{for all } i \neq j
\end{equation}

where $\bar{\rho}$ is computed as:

\begin{equation}
\bar{\rho} = \frac{2}{N(N-1)} \sum_{i<j} \rho_{ij}
\end{equation}

The covariance matrix is then constructed as:

\begin{equation}
\Sigma_{\text{CC}} = D_{\sigma}^{1/2} R_{\text{const}} D_{\sigma}^{1/2}
\end{equation}

where $D_{\sigma} = \text{diag}(\sigma_1^2, \ldots, \sigma_N^2)$ contains the individual stock variances and $R_{\text{const}}$ is the constant correlation matrix with $\bar{\rho}$ on the off-diagonal elements.

This approach is particularly effective when true correlations are relatively homogeneous, as it eliminates noise in correlation estimates while preserving differences in individual stock volatilities.

\subsubsection{Shrinkage Estimator}

The shrinkage approach combines the sample covariance matrix with a structured target matrix to reduce estimation error \citep{ledoit2003}:

\begin{equation}
\Sigma_{\text{shrink}} = \delta F + (1 - \delta) S
\end{equation}

where $S$ is the sample covariance matrix, $F$ is a structured target (we use an identity matrix scaled by average variance), and $\delta \in [0,1]$ is the shrinkage intensity.

We set $\delta = 0.5$ in our implementation, though optimal shrinkage methods that estimate $\delta$ from the data are also available \citep{ledoit2004}.

\subsection{Portfolio Optimization}

For each covariance estimation method, we solve the following optimization problem:

\begin{equation}
\begin{aligned}
\min_{w} \quad & w' \Sigma w \\
\text{s.t.} \quad & \sum_{i=1}^N w_i = 1 \\
& w_i \geq 0, \quad i = 1, \ldots, N \\
& w_i \leq u, \quad i = 1, \ldots, N \quad \text{(optional)}
\end{aligned}
\end{equation}

where $w$ is the $N \times 1$ vector of portfolio weights, $\Sigma$ is the estimated covariance matrix, and $u$ is the optional upper bound constraint (we use $u = 0.03$ for the constrained portfolios).

The long-only constraint ($w_i \geq 0$) prevents short positions, while the budget constraint ($\sum w_i = 1$) ensures the portfolio is fully invested. The upper bound constraint limits individual position sizes to prevent excessive concentration in a small number of stocks.

We solve this quadratic programming problem using the Sequential Least Squares Programming (SLSQP) method implemented in Python's SciPy optimization library.

\subsection{Benchmark Portfolios}

We compare our minimum variance portfolios against two naive diversification strategies:

\textbf{Equal-Weighted Portfolio:} All stocks receive equal weight:
\begin{equation}
w_i^{EW} = \frac{1}{N}
\end{equation}

\textbf{Value-Weighted Portfolio:} Weights are proportional to market capitalization:
\begin{equation}
w_i^{VW} = \frac{\text{MktCap}_i}{\sum_{j=1}^N \text{MktCap}_j}
\end{equation}

\subsection{Backtesting Framework}

We implement a rolling-window backtesting procedure to evaluate out-of-sample performance:

\begin{enumerate}
    \item At each month $t$ in the out-of-sample period (January 2017 to December 2022), use the previous $T = 60$ months of data to estimate covariance matrices.
    \item Compute optimal portfolio weights for each strategy using the estimated covariance matrices and the previous month's market capitalizations.
    \item Hold these weights for month $t$ and record the realized portfolio return.
    \item Calculate turnover as:
    \begin{equation}
    \text{Turnover}_t = \frac{1}{2} \sum_{i=1}^N |w_{i,t} - w_{i,t-1}|
    \end{equation}
    where $w_{i,t}$ are the target weights at time $t$ and $w_{i,t-1}$ are the actual weights after market movements from the previous period.
    \item Advance one month and repeat.
\end{enumerate}

This procedure generates a time series of out-of-sample portfolio returns for each strategy, which we use to compute performance metrics.

\subsection{Performance Metrics}

We evaluate portfolios using the following metrics:

\begin{itemize}
    \item \textbf{Annualized Return:} Geometric average return scaled to annual frequency
    \item \textbf{Annualized Volatility:} Standard deviation of returns scaled to annual frequency
    \item \textbf{Sharpe Ratio:} Excess return per unit of risk: $\text{SR} = \frac{\bar{r} - r_f}{\sigma}$
    \item \textbf{Maximum Drawdown:} Largest peak-to-trough decline in cumulative returns
    \item \textbf{Average Turnover:} Mean monthly turnover across the out-of-sample period
\end{itemize}

\section{Results}

\subsection{Performance Summary}

Table \ref{tab:performance} presents comprehensive performance statistics for all portfolio strategies over the out-of-sample period (January 2017 - December 2022). Several key findings emerge from this analysis.

% PLACEHOLDER: Insert your actual results table here
\begin{table}[H]
\centering
\caption{Out-of-Sample Performance Statistics (January 2017 - December 2022)}
\label{tab:performance}
\begin{tabular}{lcccccc}
\toprule
\textbf{Strategy} & \textbf{Total} & \textbf{Ann.} & \textbf{Ann.} & \textbf{Sharpe} & \textbf{Max} & \textbf{Avg} \\
 & \textbf{Return} & \textbf{Return} & \textbf{Vol.} & \textbf{Ratio} & \textbf{DD} & \textbf{Turnover} \\
\midrule
Sample Cov. & 140.9\% & 15.8\% & 17.3\% & 0.87 & -23.8\% & XX\% \\
Single Factor & 67.3\% & 9.0\% & 16.2\% & 0.54 & -32.6\% & XX\% \\
Constant Corr. & 107.9\% & 13.0\% & 13.5\% & 0.88 & -15.3\% & XX\% \\
Shrinkage & 145.5\% & 16.1\% & 16.3\% & 0.93 & -22.4\% & XX\% \\
Single Factor (3\%) & XX\% & XX\% & XX\% & XX & XX\% & XX\% \\
Constant Corr. (3\%) & XX\% & XX\% & XX\% & XX & XX\% & XX\% \\
Shrinkage (3\%) & XX\% & XX\% & XX\% & XX & XX\% & XX\% \\
Equal Weight & 192.4\% & 19.6\% & 17.6\% & 1.04 & -21.4\% & XX\% \\
Value Weight & 145.3\% & 16.1\% & 17.9\% & 0.86 & -25.2\% & XX\% \\
\bottomrule
\end{tabular}
\begin{tablenotesize}
\textit{Note:} Replace XX\% with actual values from your notebook output. Performance metrics are computed on monthly returns over the 72-month out-of-sample period. Sharpe ratios use the 90-day T-bill rate as the risk-free rate.
\end{tablenotesize}
\end{table}

First, minimum variance portfolios successfully achieve their primary objective of risk reduction. The constant correlation model portfolio exhibits the lowest annualized volatility at 13.5\%, substantially below both the equal-weighted (17.6\%) and value-weighted (17.9\%) benchmarks. This represents a reduction in volatility of approximately 23-24\% compared to naive diversification.

Second, the risk reduction does not necessarily come at the cost of returns. The shrinkage estimator achieves an annualized return of 16.1\%, matching the value-weighted benchmark, while delivering slightly lower volatility. This results in the highest Sharpe ratio (0.93) among the minimum variance strategies, though the equal-weighted portfolio achieves the highest overall Sharpe ratio of 1.04.

Third, the different covariance estimation methods produce meaningfully different results. The single-factor model delivers the weakest performance, with both lower returns (9.0\%) and higher volatility (16.2\%) than alternative approaches. This suggests that the single-factor model's parsimony comes at a cost in this application. In contrast, the constant correlation model achieves the best risk reduction while maintaining competitive returns.

Fourth, the sample covariance portfolio performs reasonably well, though it exhibits higher volatility than the constant correlation approach. This somewhat surprising result---given the well-documented instability of sample covariance matrices---may reflect our moderately sized universe (200-400 stocks) and relatively long estimation window (60 months).

\subsection{Cumulative Returns Analysis}

Figure \ref{fig:cumulative} displays the cumulative performance of all strategies over the out-of-sample period. Several patterns are evident.

% PLACEHOLDER: Insert your cumulative returns chart here
\begin{figure}[H]
\centering
\includegraphics[width=0.95\textwidth]{cumulative_returns.png}
\caption{Cumulative Returns for All Portfolio Strategies (January 2017 - December 2022). Dashed lines indicate portfolios with 3\% upper bound constraints.}
\label{fig:cumulative}
\end{figure}

The period from January 2017 through February 2020 represents a generally favorable environment for equities, with all strategies producing positive cumulative returns. During this period, the equal-weighted portfolio exhibits the strongest performance, consistent with the well-documented size premium. Minimum variance portfolios generally trail the benchmarks but maintain more stable trajectories with lower volatility.

The COVID-19 market crash in March 2020 provides a critical test of the minimum variance strategies. The constant correlation portfolio demonstrates superior downside protection, experiencing the smallest drawdown among all strategies. The single-factor model, however, suffers the largest decline, raising questions about its effectiveness during periods of market stress.

The recovery period from April 2020 through early 2022 sees all strategies recoup losses and reach new highs. The equal-weighted portfolio's outperformance becomes more pronounced during this period, driven by the strong performance of smaller-cap stocks.

Finally, the market decline in 2022 again highlights differences in downside protection across strategies. The constant correlation model demonstrates resilience, while the single-factor approach again exhibits larger drawdowns.

\subsection{Impact of Upper Bound Constraints}

Table \ref{tab:constraint_impact} compares the performance of minimum variance portfolios with and without the 3\% upper bound constraint.

% PLACEHOLDER: Create this comparison table from your results
\begin{table}[H]
\centering
\caption{Impact of 3\% Upper Bound Constraint on Portfolio Characteristics}
\label{tab:constraint_impact}
\begin{tabular}{lcccccc}
\toprule
 & \multicolumn{2}{c}{\textbf{Single Factor}} & \multicolumn{2}{c}{\textbf{Constant Corr.}} & \multicolumn{2}{c}{\textbf{Shrinkage}} \\
 \cmidrule(lr){2-3} \cmidrule(lr){4-5} \cmidrule(lr){6-7}
\textbf{Metric} & \textbf{No Bound} & \textbf{3\% Bound} & \textbf{No Bound} & \textbf{3\% Bound} & \textbf{No Bound} & \textbf{3\% Bound} \\
\midrule
Ann. Return & 9.0\% & XX\% & 13.0\% & XX\% & 16.1\% & XX\% \\
Ann. Volatility & 16.2\% & XX\% & 13.5\% & XX\% & 16.3\% & XX\% \\
Sharpe Ratio & 0.54 & XX & 0.88 & XX & 0.93 & XX \\
Max Drawdown & -32.6\% & XX\% & -15.3\% & XX\% & -22.4\% & XX\% \\
Avg Turnover & XX\% & XX\% & XX\% & XX\% & XX\% & XX\% \\
Max Weight & XX\% & 3.0\% & XX\% & 3.0\% & XX\% & 3.0\% \\
\bottomrule
\end{tabular}
\begin{tablenotesize}
\textit{Note:} Replace XX\% with actual values from your notebook output comparing constrained and unconstrained portfolios.
\end{tablenotesize}
\end{table}

The upper bound constraint serves several purposes. First, it prevents extreme concentration in a small number of stocks, which can occur when the optimizer identifies a subset of securities with particularly low covariances. Without constraints, some portfolios may allocate 10-20\% or more to individual stocks, creating significant idiosyncratic risk.

Second, the constraint enhances robustness to estimation error. By preventing the optimizer from taking extreme positions based on potentially spurious covariance estimates, the constraint can improve out-of-sample performance.

However, these benefits come at a cost. The constraint forces the optimizer to include more stocks than it would otherwise prefer, potentially increasing portfolio variance. Additionally, turnover typically increases because the constraint binds more frequently as covariance estimates change over time.

Our results indicate that [DESCRIBE YOUR ACTUAL RESULTS - does the constraint help or hurt? How much does turnover increase? Is the risk reduction benefit diminished?].

\subsection{Risk-Return Trade-off}

Figure \ref{fig:risk_return} presents the risk-return profile of all strategies in a scatter plot format.

% PLACEHOLDER: Insert your risk-return scatter plot here
\begin{figure}[H]
\centering
\includegraphics[width=0.8\textwidth]{risk_return_scatter.png}
\caption{Risk-Return Trade-off for All Portfolio Strategies}
\label{fig:risk_return}
\end{figure}

This visualization clearly demonstrates the minimum variance portfolios' positioning on the efficient frontier. The constant correlation model achieves the leftmost position (lowest risk), consistent with its design objective. The shrinkage estimator offers an attractive middle ground, providing substantial risk reduction while maintaining returns comparable to the market-cap-weighted benchmark.

The equal-weighted portfolio occupies the upper-right region, delivering the highest returns but at the cost of elevated volatility. This pattern reflects the size and value premiums captured by equal weighting during the sample period.

\subsection{Turnover Analysis}

Figure \ref{fig:turnover} compares average monthly turnover across strategies.

% PLACEHOLDER: Insert your turnover comparison chart here
\begin{figure}[H]
\centering
\includegraphics[width=0.8\textwidth]{turnover_comparison.png}
\caption{Average Monthly Turnover by Portfolio Strategy}
\label{fig:turnover}
\end{figure}

Turnover is a critical practical consideration, as it directly translates to transaction costs. High turnover can substantially erode the theoretical benefits of sophisticated portfolio construction techniques.

[DESCRIBE YOUR ACTUAL RESULTS - which strategies have highest/lowest turnover? Does the 3\% constraint increase turnover? Are the turnover levels economically significant?]

The benchmark portfolios exhibit relatively low turnover. The value-weighted portfolio's turnover reflects only changes in relative market capitalizations, while the equal-weighted portfolio rebalances to maintain equal weights.

\section{Discussion}

\subsection{Covariance Estimation Methods}

Our empirical analysis reveals important differences in the effectiveness of alternative covariance estimation approaches for minimum variance portfolio construction.

The \textbf{constant correlation model} emerges as the most effective risk reduction tool, achieving the lowest volatility among all strategies. This result aligns with \cite{elton1973}'s original proposition that the constant correlation model performs well when correlations are relatively homogeneous. The model's success likely reflects two factors: (1) it eliminates noise in individual correlation estimates while preserving information about heterogeneous volatilities, and (2) equity return correlations tend to cluster around a common level, making the equal-correlation assumption reasonably accurate.

The \textbf{shrinkage estimator} delivers the best risk-adjusted performance among minimum variance strategies. By combining the sample covariance matrix with a structured target, shrinkage balances the bias-variance trade-off more effectively than either extreme. This middle ground proves valuable in our application, where the sample size (60 months) provides reasonable but not overwhelming statistical power.

The \textbf{single-factor model's} disappointing performance warrants explanation. While its parsimony should theoretically reduce estimation error, the model may be too restrictive for our universe. The single-factor framework assumes all covariances derive solely from common exposure to the market factor, ignoring industry effects, style factors, and other sources of common variation. This oversimplification appears costly in our application.

Interestingly, the \textbf{sample covariance matrix} performs reasonably well despite theoretical concerns about its instability. This may reflect our moderately sized universe (typically 200-400 stocks) and relatively long estimation window (60 months), which provide sufficient observations per parameter to avoid the most severe estimation problems.

\subsection{Risk Reduction Effectiveness}

Minimum variance portfolios successfully achieve their primary objective: reducing portfolio volatility relative to naive diversification benchmarks. The constant correlation portfolio reduces volatility by approximately 23-24\% compared to equal-weighted and value-weighted benchmarks.

This risk reduction proves particularly valuable during market downturns. The COVID-19 crash in March 2020 and the market decline in 2022 both demonstrate the defensive characteristics of minimum variance portfolios. The constant correlation strategy, in particular, exhibits maximum drawdowns substantially smaller than benchmark portfolios.

However, risk reduction does not guarantee superior risk-adjusted returns. While the shrinkage estimator achieves a competitive Sharpe ratio, the equal-weighted portfolio ultimately delivers the highest Sharpe ratio in our sample. This reflects the favorable performance of smaller-cap stocks during the 2017-2022 period, which benefits the equal-weighted approach.

These results highlight an important distinction: minimum variance portfolios aim to minimize risk, not maximize risk-adjusted returns. For investors with strong risk aversion or explicit volatility targets, minimum variance strategies offer clear benefits. For investors focused on risk-adjusted returns, the case is more nuanced and depends on the specific market environment.

\subsection{Upper Bound Constraints}

The 3\% upper bound constraint represents a practical tool for managing portfolio concentration. [INTERPRET YOUR ACTUAL RESULTS HERE - Based on Table 2, discuss whether the constraint helps or hurts performance, how it affects diversification, and the turnover implications.]

From a theoretical perspective, the constraint forces the portfolio away from the unconstrained minimum variance point on the efficient frontier, necessarily increasing portfolio variance (assuming the constraint binds). However, this cost may be outweighed by benefits including: (1) reduced exposure to estimation error when the unconstrained optimizer wants to take extreme positions, (2) enhanced diversification of idiosyncratic risks, and (3) improved alignment with regulatory requirements or institutional constraints on position sizing.

The turnover implications deserve particular attention. If the 3\% constraint binds frequently and the binding stocks change over time, the constraint can substantially increase trading costs. In our analysis, [DESCRIBE WHETHER THIS OCCURS IN YOUR RESULTS].

\subsection{Practical Implementation Considerations}

Several practical factors merit consideration when implementing minimum variance portfolios:

\textbf{Transaction Costs:} Even moderate turnover can erode returns through commissions, bid-ask spreads, and market impact. For large portfolios or less liquid securities, transaction costs could eliminate much of the theoretical benefit. Our turnover analysis suggests [SUMMARIZE YOUR TURNOVER FINDINGS].

\textbf{Estimation Frequency:} Our monthly rebalancing schedule represents a design choice. Less frequent rebalancing (e.g., quarterly) would reduce turnover and transaction costs but might fail to adapt quickly to changing market conditions. More frequent rebalancing (e.g., daily) could improve responsiveness but would likely prove prohibitively expensive.

\textbf{Universe Definition:} Our approach of including all available stocks with sufficient data creates a dynamic universe that changes over time. An alternative approach would fix the universe at the start of the sample period, ensuring consistent composition but potentially including delisted firms or excluding new entrants.

\textbf{Parameter Choices:} Several methodological choices affect results, including the estimation window length (we use 60 months), the shrinkage intensity (we use $\delta = 0.5$), and the upper bound level (we use 3\%). Sensitivity analysis could reveal whether our results are robust to alternative specifications.

\subsection{Performance Across Market Conditions}

The 2017-2022 sample period encompasses diverse market environments, providing a robust test of portfolio strategies.

The \textbf{bull market period (2017-2019)} favors growth-oriented strategies. Equal weighting's exposure to smaller-cap stocks drives outperformance, while minimum variance portfolios' defensive positioning leads them to lag. This pattern is expected---minimum variance is a defensive strategy that sacrifices upside participation for downside protection.

The \textbf{COVID-19 crash (March 2020)} dramatically illustrates the value of risk reduction. The constant correlation portfolio experiences the smallest decline, validating its defensive characteristics. The recovery's speed somewhat limits this advantage, as all portfolios quickly recoup losses, but the interim drawdown differences are meaningful for risk-averse investors.

The \textbf{recovery and growth period (2020-2021)} again favors equal weighting, which benefits from the strong performance of pandemic winners and smaller-cap stocks. Minimum variance portfolios participate in the rally but lag the more aggressive benchmarks.

The \textbf{2022 market decline} provides another test of defensive strategies. Rising interest rates and inflation concerns create a challenging environment for equities. [DESCRIBE HOW DIFFERENT STRATEGIES PERFORM IN YOUR DATA DURING THIS PERIOD].

\section{Conclusion}

This paper provides a comprehensive empirical evaluation of minimum variance portfolio construction methods using recent U.S. equity market data. Our analysis yields several key conclusions with practical implications for portfolio management.

\subsection{Key Findings}

\textbf{First}, minimum variance portfolios successfully achieve their primary objective of risk reduction. The constant correlation model, in particular, delivers volatility approximately 23-24\% below naive diversification benchmarks. This risk reduction translates to smaller maximum drawdowns during market stress periods, including the COVID-19 crash and 2022 market decline.

\textbf{Second}, the choice of covariance estimation method materially affects portfolio performance. The constant correlation model proves most effective for pure risk reduction, while the shrinkage estimator offers the best balance of risk and return among minimum variance strategies. The single-factor model's disappointing performance suggests that its parsimony may be excessive for this application.

\textbf{Third}, the 3\% upper bound constraint [SUMMARIZE YOUR FINDINGS ON WHETHER THE CONSTRAINT HELPS OR HURTS, AND THE TURNOVER IMPLICATIONS]. This illustrates the tension between theoretical optimality and practical considerations in portfolio construction.

\textbf{Fourth}, minimum variance portfolios demonstrate time-varying effectiveness across different market regimes. They provide particular value during market downturns but may lag during strong bull markets. This pattern suggests these strategies are most appropriate for risk-averse investors willing to sacrifice some upside potential for enhanced downside protection.

\subsection{Implications for Practice}

For practitioners considering minimum variance strategies, our results suggest several recommendations:

\begin{itemize}
    \item The constant correlation model merits consideration for investors prioritizing risk reduction above all else.
    \item The shrinkage estimator offers an attractive middle ground for investors seeking risk reduction without sacrificing too much return potential.
    \item Upper bound constraints should be calibrated based on the specific application, recognizing the trade-off between enhanced diversification and increased turnover.
    \item Transaction costs must be carefully considered, as even moderate turnover can erode the theoretical benefits of sophisticated optimization.
    \item Minimum variance strategies are best suited for risk-averse investors with long time horizons who can tolerate underperformance during strong bull markets in exchange for superior downside protection.
\end{itemize}

\subsection{Limitations and Future Research}

Our analysis is subject to several limitations that suggest directions for future research.

First, our sample period, while including significant market stress events, represents only six years of out-of-sample data. Longer sample periods would strengthen confidence in the results and enable analysis of performance across complete market cycles.

Second, we examine a single universe definition and rebalancing frequency. Alternative specifications---such as restricting the universe to large-cap stocks or implementing quarterly rebalancing---might yield different results.

Third, we do not explicitly model transaction costs, instead using turnover as a proxy for trading expenses. A more complete analysis would incorporate realistic cost models and potentially optimize the rebalancing frequency to balance trading costs against portfolio optimality.

Fourth, we consider only three covariance estimation methods. Other approaches, including factor models beyond the single-factor specification, could be explored. Machine learning methods for covariance estimation represent a particularly promising direction for future research.

Finally, our focus on minimum variance portfolios excludes consideration of expected returns. Extensions incorporating return forecasts while maintaining attention to risk could offer additional insights.

Despite these limitations, our analysis provides valuable evidence on the practical effectiveness of minimum variance portfolios and highlights important considerations for their implementation. As investors continue to seek strategies offering protection against market volatility, minimum variance approaches represent a theoretically sound and empirically validated tool for portfolio management.

\bibliographystyle{plainnat}
\begin{thebibliography}{9}

\bibitem[Clarke et al.(2006)]{clarke2006}
Clarke, R., de Silva, H., \& Thorley, S. (2006).
\newblock Minimum-variance portfolios in the U.S. equity market.
\newblock \textit{Journal of Portfolio Management}, 33(1), 10--24.

\bibitem[Elton \& Gruber(1973)]{elton1973}
Elton, E. J., \& Gruber, M. J. (1973).
\newblock Estimating the dependence structure of share prices---implications for portfolio selection.
\newblock \textit{Journal of Finance}, 28(5), 1203--1232.

\bibitem[Jobson \& Korkie(1980)]{jobson1980}
Jobson, J. D., \& Korkie, B. (1980).
\newblock Estimation for Markowitz efficient portfolios.
\newblock \textit{Journal of the American Statistical Association}, 75(371), 544--554.

\bibitem[Ledoit \& Wolf(2003)]{ledoit2003}
Ledoit, O., \& Wolf, M. (2003).
\newblock Improved estimation of the covariance matrix of stock returns with an application to portfolio selection.
\newblock \textit{Journal of Empirical Finance}, 10(5), 603--621.

\bibitem[Ledoit \& Wolf(2004)]{ledoit2004}
Ledoit, O., \& Wolf, M. (2004).
\newblock Honey, I shrunk the sample covariance matrix.
\newblock \textit{Journal of Portfolio Management}, 30(4), 110--119.

\bibitem[Markowitz(1952)]{markowitz1952}
Markowitz, H. (1952).
\newblock Portfolio selection.
\newblock \textit{Journal of Finance}, 7(1), 77--91.

\bibitem[Merton(1980)]{merton1980}
Merton, R. C. (1980).
\newblock On estimating the expected return on the market: An exploratory investigation.
\newblock \textit{Journal of Financial Economics}, 8(4), 323--361.

\end{thebibliography}

\newpage
\appendix
\section{Additional Figures and Tables}

% PLACEHOLDER: Add any additional charts you want to include
\begin{figure}[H]
\centering
\includegraphics[width=0.9\textwidth]{rolling_volatility.png}
\caption{Rolling 12-Month Volatility by Portfolio Strategy}
\label{fig:rolling_vol}
\end{figure}

\begin{figure}[H]
\centering
\includegraphics[width=0.9\textwidth]{drawdown_analysis.png}
\caption{Portfolio Drawdowns Over Time}
\label{fig:drawdowns}
\end{figure}

% PLACEHOLDER: Add weight concentration analysis if desired
% \begin{table}[H]
% \centering
% \caption{Portfolio Concentration Statistics}
% \label{tab:concentration}
% ...
% \end{table}

\end{document}
